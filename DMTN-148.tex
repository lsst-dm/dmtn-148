\documentclass[DM,authoryear,toc]{lsstdoc}
% lsstdoc documentation: https://lsst-texmf.lsst.io/lsstdoc.html
\input{meta}

% Package imports go here.

% Local commands go here.

%If you want glossaries
%\input{aglossary.tex}
%\makeglossaries

\title{DM Calibration Products}

% Optional subtitle
% \setDocSubtitle{A subtitle}

\author{%
Christopher Z Waters
}

\setDocRef{DMTN-148}
\setDocUpstreamLocation{\url{https://github.com/lsst-dm/dmtn-148}}

\date{\vcsDate}

% Optional: name of the document's curator
% \setDocCurator{The Curator of this Document}

\setDocAbstract{%
The goal of this tech note is to provide a plan for the handling of calibration products in the Gen-3 Butler environment, including the construction, validation, certification, and deprecation.  This process is equally challenging with the current Gen-2 Butlers, but the hope is that with a clear goal in mind, the currently existing calibrations, for all calibration types and cameras, can be migrated to a common design that will be shared with Gen-3.  The current state, the end goal, and the transition steps between these are defined.
}

% Change history defined here.
% Order: oldest first.
% Fields: VERSION, DATE, DESCRIPTION, OWNER NAME.
% See LPM-51 for version number policy.
\setDocChangeRecord{%
  \addtohist{1}{2020-04-16}{Initial version.}{Christopher Waters}
  \addtohist{2}{2020-12-31}{Updates to represent how calibrations were actually made.}{Christopher Waters}
}


\begin{document}

% Create the title page.
\maketitle
% Frequently for a technote we do not want a title page  uncomment this to remove the title page and changelog.
% use \mkshorttitle to remove the extra pages

% ADD CONTENT HERE
% You can also use the \input command to include several content files.

\section{Intro}

The goal of this tech note is to provide a plan for the handling of
calibration products in the Gen-3 Butler environment.  There are a
large number of these products currently in use, but many of them do
not have a clearly defined process to define, ingest, test, and
deprecate them.  The lack of documentation on how these processes work
has led to many implementations of similar paths, further complicating
the issue.


\section{The End State}

The final state that we wish to get to has the following properties:

\begin{itemize}
\item All calibration products are either lsst.afw.image.Exposure
  image data or a subclass of the \verb|CalibType| calibration class
  from the \verb|ip_isr| package.
%% CZW: input data => raw images.
%%      single vs multiple pipelines.
%%      provenance as a butler exportable object.
\item As much as possible, all calibration products should be
  constructed by a DM Task in the \verb|cp_pipe| package.  The raw
  input data used for the construction will be available from the
  Gen-3 Butler, and can be exported along with the task configuration
  parameters to allow these calibrations to be reconstructed in the
  future.  Exceptions will exist for external or lab created
  calibrations, although we expect the number of these calibrations to
  be small.
\item The provenance, config, and final product files may be saved to
  an \verb|obs_CAMERA_data| package in git for convenience.  This
  allows these files to be directly migrated to other Butler
  repositories without being reconstructed.  Making these products
  human readable allows the tracking of value changes with the
  evolution of the set of exported products.
%% CZW: Prior calibrations used for live raw data.
%%      cp_verify to show that live calibrations are consistent with that.
\item All calibration products will be verified prior to
  certification.  The goal of this process is to demonstrate that the
  newly constructed calibration satisfies a number of quantifiable
  quality checks.  This should reduce the amount of manual image
  examination necessary for certification.  DMTN-101 will define the
  exact tests for each calibration type.
\item All calibration products are certified for use and have the
  valid date range set manually, preventing test runs from being used
  inadvertently.  Calibrations ingested from an \verb|obs_CAMERA_data|
  package in git should have sufficient metadata that supply
  appropriate date ranges.
\item The verification pipeline will also be run on new raw
  calibration data.  This will ensure that the certified calibrations
  being used for the live raw data stream are appropriate, and to
  monitor the variations of the defined metrics with time.
\item Calibration information stored in a lsst.afw.cameraGeom object
  may be used, but is always assumed to be overridden by supplied
  calibration products.  No new calibration information should be
  stored in these objects, and currently existing calibration
  information should be deprecated.

\end{itemize}

\section{The Current State}

\subsection{Standard Calibrations}

%% CZW: collection namning consistent with RFC-741/DMTN-167
The classic calibration products, \verb|BIAS|, \verb|DARK|,
\verb|FLAT|, and \verb|FRINGE|, all write FITS files that are indexed
in Butler's calibRegistry.sqlite3 repository database.  These files
are created by \verb|pipe_drivers|/constructCalibs.py, and are
ingested into the repository upon acceptance by
\verb|pipe_tasks|/ingestCalibs.py.  This ingestion process sets the
``validStart'' and ``validEnd'' values to restrict the usage of the
calibration, as well as some identifying string which is used (along
with calibration type, filter, and detector) to uniquely identify the
calibration on disk.  We have been using the ``calibDate'', indicating
when the calibration was constructed or ingested, as this string.  For
calibrations that are generated by \verb|cp_pipe|, it may be better to
use the Butler collection that the calibration was generated in (which
will enforce uniqueness on simultaneous processing).  Calibrations
that are ingested from git could use the repository SHA1 hash, which
would provide similar uniqueness.  See the discussion below for more
on Gen3 collection naming.

Although not formally a ``classic'' calibration product, the \verb|SKY|
calibration is produced, stored, and accessed the same way.

\subsection{``User-curated'' Calibrations}

The other inputs necessary for ISR have generally been described as
"user-curated".  These products are either defined by the \verb|CAMERA| as
part of the afw.cameraGeom description of the instrument, or are
variously persisted and accessed in unique ways.  Describing a path to
fixing this is a goal of this tech note.

As part of the \verb|DEFECTS| rework (RFC-595), there is now a fixed and
generic way to handle the ingest, if not necessarily the construction,
of that product.  The \verb|QE_CURVE| (RFC-625) has been implemented
following this process.  These \verb|DEFECTS|-like calibration products have
their origins as human-curated text files in the appropriate
\verb|obs_CAMERA_data| packages (and are thereby versioned through the
packages' git repositories).  The text file must be in a structured
format, to allow the header keywords to be read and assigned to a
metadata PropertyList.

For these products to be used, they are converted into a FITS format
that can be read by ingestCalibs.py.  This is performed by the
\verb|pipe_tasks|/ingestCuratedCalibs.py script, which uses the
metadata information and the calibration product's python class to
read the text version and convert to a FITS file format that can be
ingested by ingestCalibs.py.

This updated calibration process removes the necessity that each
\verb|obs_CAMERA_data| package implement one-off mapper methods to
allow the product to be found.  However, only \verb|DEFECTS|,
\verb|QE_CURVE|, and \verb|LINEARITY| follow this system, leaving the
remaining products to be handled with camera-specific code and
cameraGeom entries.  In addition to having diverse implementations,
these handlers are often rigid, and cannot be versioned easily.

\section{Transitioning to Gen3/What We Should Expect From Gen3}

\subsection{The Gen3 Workflow}

Figure \ref{fig:StandardWF} illustrates the expected standard workflow.
Raw calibration data will be processed through a pipeline defined in
the \verb|cp_pipe| package, which will generate a new master
calibration.  This calibration will be constructed into its own Butler
collection, and will not be generally usable.  The calibration will be
processed through a validation pipeline (possibly depending on
additional raw data), producing a set of metrics that illustrate that
the calibration satisfies its purpose.  Additional processing with the
proposed calibration may then be performed to satisfy to a designated
``expert'' that the calibration should be accepted.  The expert then
certifies the validity range the calibration is to be used for, and
assigns it to a collection of recommended calibrations.

It may be useful to export the calibration products that are
processing intensive to the appropriate \verb|obs_*_data| package.
This export process will retain the certification dates, allowing
these products to be directly imported into other Butler instances,
skipping the intensive processing phase, as illustrated by
\ref{fig:ExternalWF}.  This procedure can also be used for calibrations
that the DM team is not planning to be able to generate, such as
detailed quantum efficiency curves.  Adding these calibrations to the
\verb|obs_*_data| package will allow them to be formatted
appropriately with the relevant metadata, including the validity
range, expected for import.  These external calibration sources are
discussed further below as ``curated calibrations''.

Figure \ref{fig:EditedWF} illustrates how this process can also be used
for manual editing of the calibration.  An example of this is the
addition of entries to a \verb|DEFECTS| calibration during the
inspection as part of certification.  The final edited version of this
calibration will be added to the \verb|obs_*_data| package, along with
the metadata and validity range for this calibration, and that can
then be re-imported to the Butler and certified for use.

This workflow will also use the Butler collection naming conventions
described in DMTN-167.  JIRA tickets will be used to coordinate the
production of new calibrations, and so those ticket numbers will be
used in the production collection, with
\verb|<instrument>/calib/<ticket>/*| containing the production runs
for the calibrations, and \verb|<instrument>/calib/<ticket>| being the
calibration collection those calibrations are certified to.  These
calibration collections will then be connected into a Butler
``chained'' collection named \verb|<instrument>/calib| that contains
the recommended calibrations for that instrument.

\begin{figure}
  \caption{Standard calibration creation. \label{fig:StandardWF}}
  \centering
  \includegraphics[width=0.9\textwidth]{figures/Standard_Calibration.pdf}
\end{figure}

\begin{figure}
  \caption{External calibration ingest. \label{fig:ExternalWF}}
  \centering
  \includegraphics[width=0.9\textwidth]{figures/External_Calibration.pdf}
\end{figure}

\begin{figure}
  \caption{Hand-Edited calibration creation. \label{fig:EditedWF}}
  \centering
  \includegraphics[width=0.9\textwidth]{figures/Hand-Edited_Calibration.pdf}
\end{figure}


\subsection{The IsrCalib Class}
The first step to a consistent calibration system is to ensure that
all products more complex than a simple image use the new
\verb|IsrCalib| base class.  This includes all calibrations in both
Gen-2 and Gen-3, to ensure that we can use the same calibration data.

The default storage format for calibration products will be FITS, as
that takes advantage of the existing Butler formatter, and matches the
expectation for the image based calibrations.  A text format should
exist for as many calibration types as possible, to allow for a
human-readable form to be stored as a curated calibration in the
\verb|obs_*_data| packages.

The \verb|IsrCalib| is an abstract base class that can be subclassed to implement methods that handle the storage and application of the calibation, as listed in Table \ref{tab:IsrCalib}.
\begin{tabular}{l l}
  \label{tab:IsrCalib}
  Method & Type & Description \\
  \hline
  fromDetector & Optional & Used to construct a calibration from the detector cameraGeom. \\
  fromDict & Required & Construct a calibration from a dictionary of properties. \\
  toDict & Required & Return a dictionary contining the calibration properties. \\
  fromTable & Required & Construct a calibration from a list of astropy tables. \\
  toTable & Required & Return a list of astropy tables containing the calibration properties. \\
  validate & Optional & Validate the calibration is appropriate for the image to process. \\
  apply & Optional & Apply the calibration to the image to process. \\
\end{tabular}

\subsection{Metadata and Header Keywords}

The required header keywords are currently poorly standarized.  The
\verb|OBSTYPE| keyword is universally populated in all calibrations,
regardless of if they are generated by constructCalibs.py or are
curated by the \verb|obs_*_data| packages.  Curated calibrations also
require the \verb|DETECTOR| and \verb|CALIBDATE|, the last of which is
also populated by constructCalibs.py.  The \verb|CALIB_ID| is set by
constructCalibs.py, and exists for the majority of curated
calibrations, but is not consistently used.

The keywords listed in table \ref{tab:metadata} are the proposed list
of header keywords that will be required in calibrations in Gen3.  To
conform with the expected changes to FITS persistence in RFC-686,
these and all other calibration-specific metadata fields will be
upper-case.

\begin{tabular}{l l}
  \label{tab:metadata}
  Keyword & Description \\
  \hline
  %% CZW: detector is an integer.
  %%      calib_id contents
  %%      split creation date/time still necessary?
  %%      validity range is a butler option.
  \multicolumn{2}{c}{Keywords available as calibration properties.}
  \verb|OBSTYPE | & Calibration dataset type contained in this file. \\
  \verb|{OBSTYPE}_VERSION| & Storage schema version, to allow backwards compatibility. \\
  \verb|INSTRUME| & Instrument/camera this calibration belongs to. \\
  \verb|RAFTNAME| & Raft containing the detector. \\
  \verb|SLOTNAME| & Slot in the raft containing the detector. \\
  \verb|DETECTOR| & Integer identifier for the detector. \\
  \verb|DET_NAME| & String identifier for the detector. \\
  \hline
  \multicolumn{2}{c}{Keywords set at creation time.}
  \verb|CALIB_CREATION_DATE| & Creation date of the calibration. \\
  \verb|CALIB_CREATION_TIME| & Creation time of the calibration. \\
  \hline
  \multicolumn{2}{c}{Keywords set at creation time from all used inputs.}
  \verb|CPP_INPUT_N| & Input exposure number. \\
  \verb|CPP_INPUT_DATE_N| & Input observation date. \\
  \verb|CPP_INPUT_EXPT_N| & Input exposure time. \\
  \verb|CPP_INPUT_SCALE_N| & Input scale factor used in combination. \\
  \hline
  \multicolumn{2}{c}{Keywords that are set on export from Butler.}
  \verb|CALIBDATE| & As used, is the validity start date for the calibration. \\
  \verb|VALIDSTART| & The start date of the validity range for the calibration. \\
  \verb|VALIDEND| & The end date of the validity range for the calibration. \\
\end{tabular}

% XYZ: UGH.
\subsection{Calibration Construction in Gen3}

Calibration products are constructed using PipelineTask code defined
in the \verb|cp_pipe| package.  This package also defines the pipeline
definition that connect the new PipelineTasks with those already
existing (largely those in \verb|ip_isr|).  The goal of this is to
minimize the distinction between ``calibrations'' and ``user-curated
calibrations'' as much as possible, by generating the majority of the
calibration products directly from raw images.

The Gen-3 Butler retains a record of all inputs used in the construction of a calibration, and this record will be exported along with the calibration if it is added t
In addition to the calibration product, the PipelineTask code must
persist a record of the input data used for construction.  This, along
with the already persisted pipeline definition and task configuration,
allows the calibration to be regenerated by any user, making the
construction process much more transparent than it currently is.  A
YAML file containing the dataId parameters for all inputs would
satisfy this.  Appending additional construction statistics (for
example, weight factors applied to each input during the combination
stage) would be helpful in the validation process.

% CZW: Define OIDF = CALIB_ID
Newly generated calibration products will belong to a Gen3 collection,
and that collection will be used to manage the calibration through the
validation and certification process.  This collection may only
contain one entry for a given OIDF set.

\subsection{Validation}

Calibration products are not currently validated, largely due to the
opaque construction process.  Some set of proper validation tasks will
be defined, applying the newly constructed product to a set of test
exposures prior to that calibration product's certification.  It is
expected that these test exposures will contain the inputs used in the
calibration's construction, as well as similar exposures held back
from the construction.  Applying these validation tasks continuously
as new raw calibration frames are taken will inform the date ranges
for which the product is certified.  DMTN-101 will discuss this
validation further.

\subsection{Certification}

% CZW: File format cannot change on registration.
%      Handling validation ranges.
%      Header update on export?
%      Source of truth => gen3.  obs_camera_data => exportability.
Once a calibration product has been confirmed through validation to
correct the instrument effect it is designed for, it will be certified
with a date range denoting when the calibration is valid, and assigned
to a particular calibration collection (which may contain calibrations
the same \verb|CALIB_ID| values, as the validity range adds another
dimension).  For calibrations created by PipelineTasks, this is
equivalent to a database operation; no persisted data will change.

For calibrations ingested from an \verb|obs_*_data| package, the
validity range must be included in the package.  Along with the provenance information discussed above, the calibrations These exports should also include the configuration and
input data records.

\subsection{Modifying Calibrations}

Most calibration products need user intervention only to launch the
construction and to validate and certify the result.  However, it is
assumed that some of the products used (such as \verb|CAMERA| and
\verb|QE_CURVE| definitions) will not have a PipelineTask construction
process.  The version of these products stored in the
\verb|obs_CAMERA_data| (or \verb|obs_CAMERA| for the case of the
\verb|CAMERA| calibration) repository will be ingested into the Butler
repository as needed.  These products should still have as much
configuration and provenance data as is possible, but should also note
any manual edits with a name and date.  Enforcing the inclusion of the
\verb|obs_CAMERA_data| packages git repository SHA1 entry into the
metadata of calibration products that have been ingested from those
packages should provide sufficient information to reproduce or revert
the manual edits.  This provenance method should also be used for
calibrations that need to be augmented in non-algorithmic ways, such
as \verb|DEFECTS|.

\subsection{Deprecating Calibrations}

Calibrations that should no longer be used (due to code improvements
or newly discovered problems) can be deprecated in Gen3 with
operations on the calibration collection that contains them.  This
will require tools for collection management, but this will be
entirely Butler database operations and will likely not be calibration
specific.


\section{Conclusion}


\appendix
% Include all the relevant bib files.
% https://lsst-texmf.lsst.io/lsstdoc.html#bibliographies

\begin{tabular}{l |c|c|l|l}
  Calibration product & Gen-2 Form & Gen-3 Form & Notes \\
  \hline
  \verb|BIAS| & & & \\
  \verb|DARK| & & & \\
  \verb|FLAT| & & & \\
  \verb|FRINGE| & & & \\
  \verb|SKY| & & & \\
  \verb|CAMERA| & & & \\
  \verb|DEFECTS| & & & \\
  \verb|CROSSTALK| & & & \\
  \verb|PTC| & & & \\
  \verb|LINEARITY| & & & \\
  \verb|BFKERNEL| & & & \\
  \verb|QE_CURVE| & & & \\
  \verb|STRAYLIGHT| & & & \\
  \verb|ILLUMINATION| & & & \\
\end{tabular}

\section{References} \label{sec:bib}
\bibliography{local,lsst,lsst-dm,refs_ads,refs,books}

% Make sure lsst-texmf/bin/generateAcronyms.py is in your path
\section{Acronyms} \label{sec:acronyms}
\addtocounter{table}{-1}
\begin{longtable}{p{0.145\textwidth}p{0.8\textwidth}}\hline
\textbf{Acronym} & \textbf{Description}  \\\hline

DM & Data Management \\\hline
DMTN & DM Technical Note \\\hline
FITS & Flexible Image Transport System \\\hline
ISR & Instrument Signal Removal \\\hline
OIDF & The set of (\verb|OBSTYPE|, \verb|INSTRUME|, \verb|DETECTOR|, \verb|FILTER|). \\\hline
RFC & Request For Comment \\\hline
YAML & Yet Another Markup Language \\\hline
\end{longtable}

% If you want glossary uncomment below -- comment out the two lines above
%\printglossaries





\end{document}
